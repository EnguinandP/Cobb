%%
%% This is file `sample-manuscript.tex',
%% generated with the docstrip utility.
%%
%% The original source files were:
%%
%% samples.dtx  (with options: `manuscript')
%%
%% IMPORTANT NOTICE:
%%
%% For the copyright see the source file.
%%
%% Any modified versions of this file must be renamed
%% with new filenames distinct from sample-manuscript.tex.
%%
%% For distribution of the original source see the terms
%% for copying and modification in the file samples.dtx.
%%
%% This generated file may be distributed as long as the
%% original source files, as listed above, are part of the
%% same distribution. (The sources need not necessarily be
%% in the same archive or directory.)
%%
%% Commands for TeXCount
%TC:macro \cite [option:text,text]
%TC:macro \citep [option:text,text]
%TC:macro \citet [option:text,text]
%TC:envir table 0 1
%TC:envir table* 0 1
%TC:envir tabular [ignore] word
%TC:envir displaymath 0 word
%TC:envir math 0 word
%TC:envir comment 0 0
%%
%%
%% The first command in your LaTeX source must be the \documentclass command.
%%%% Small single column format, used for CIE, CSUR, DTRAP, JACM, JDIQ, JEA, JERIC, JETC, PACMCGIT, TAAS, TACCESS, TACO, TALG, TALLIP (formerly TALIP), TCPS, TDSCI, TEAC, TECS, TELO, THRI, TIIS, TIOT, TISSEC, TIST, TKDD, TMIS, TOCE, TOCHI, TOCL, TOCS, TOCT, TODAES, TODS, TOIS, TOIT, TOMACS, TOMM (formerly TOMCCAP), TOMPECS, TOMS, TOPC, TOPLAS, TOPS, TOS, TOSEM, TOSN, TQC, TRETS, TSAS, TSC, TSLP, TWEB.
% \documentclass[acmsmall]{acmart}

%%%% Large single column format, used for IMWUT, JOCCH, PACMPL, POMACS, TAP, PACMHCI
% \documentclass[acmlarge,screen]{acmart}

%%%% Large double column format, used for TOG
% \documentclass[acmtog, authorversion]{acmart}

%%%% Generic manuscript mode, required for submission
%%%% and peer review
\documentclass[review, sigplan]{acmart}
%% Fonts used in the template cannot be substituted; margin
%% adjustments are not allowed.
%%
%% \BibTeX command to typeset BibTeX logo in the docs
%\AtBeginDocument{%
%  \providecommand\BibTeX{{%
%    \normalfont B\kern-0.5em{\scshape i\kern-0.25em b}\kern-0.8em\TeX}}}

%% Rights management information.  This information is sent to you
%% when you complete the rights form.  These commands have SAMPLE
%% values in them; it is your responsibility as an author to replace
%% the commands and values with those provided to you when you
%% complete the rights form.
%\setcopyright{acmcopyright}
%\copyrightyear{2018}
%\acmYear{2018}
%\acmDOI{XXXXXXX.XXXXXXX}

%% These commands are for a PROCEEDINGS abstract or paper.
%\acmConference[Conference acronym 'XX]{Make sure to enter the correct
%  conference title from your rights confirmation emai}{June 03--05,
%  2018}{Woodstock, NY}
%
%  Uncomment \acmBooktitle if th title of the proceedings is different
%  from ``Proceedings of ...''!
%
%\acmBooktitle{Woodstock '18: ACM Symposium on Neural Gaze Detection,
% June 03--05, 2018, Woodstock, NY}
%\acmPrice{15.00}
%\acmISBN{978-1-4503-XXXX-X/18/06}


%%
%% Submission ID.
%% Use this when submitting an article to a sponsored event. You'll
%% receive a unique submission ID from the organizers
%% of the event, and this ID should be used as the parameter to this command.
%%\acmSubmissionID{123-A56-BU3}

%%
%% For managing citations, it is recommended to use bibliography
%% files in BibTeX format.
%%
%% You can then either use BibTeX with the ACM-Reference-Format style,
%% or BibLaTeX with the acmnumeric or acmauthoryear sytles, that include
%% support for advanced citation of software artefact from the
%% biblatex-software package, also separately available on CTAN.
%%
%% Look at the sample-*-biblatex.tex files for templates showcasing
%% the biblatex styles.
%%

%%
%% The majority of ACM publications use numbered citations and
%% references.  The command \citestyle{authoryear} switches to the
%% "author year" style.
%%
%% If you are preparing content for an event
%% sponsored by ACM SIGGRAPH, you must use the "author year" style of
%% citations and references.
%% Uncommenting
%% the next command will enable that style.
%%\citestyle{acmauthoryear}

%%
%% end of the preamble, start of the body of the document source.
\begin{document}

%%
%% The "title" command has an optional parameter,
%% allowing the author to define a "short title" to be used in page headers.
\title{Cobb: Synthesis of Test Input Generators with Coverage Types}

%%
%% The "author" command and its associated commands are used to define
%% the authors and their affiliations.
%% Of note is the shared affiliation of the first two authors, and the
%% "authornote" and "authornotemark" commands
%% used to denote shared contribution to the research.
\author{Patrick LaFontaine}
\author{Anxhelo Xhebraj}
\author{David Deng}
%\email{}
%\affiliation{%
%    \institution{Purdue University}
%    \streetaddress{}
%    \city{}
%    \state{}
%    \country{USA}
%    \postcode{}
%}


%%
%% By default, the full list of authors will be used in the page
%% headers. Often, this list is too long, and will overlap
%% other information printed in the page headers. This command allows
%% the author to define a more concise list
%% of authors' names for this purpose.
\renewcommand{\shortauthors}{LaFontaine et al.}


%\begin{abstract}
%    This thing is cool
%\end{abstract}

%%
%% The code below is generated by the tool at http://dl.acm.org/ccs.cfm.
%% Please copy and paste the code instead of the example below.
%%
%\begin{CCSXML}

%\end{CCSXML}



%%
%% Keywords. The author(s) should pick words that accurately describe
%% the work being presented. Separate the keywords with commas.
%\keywords{Do, Not, Us, This, Code, Put, the, Correct, Terms, for, Your, Paper}


%\received{20 February 2007}
%\received[revised]{12 March 2009}
%\received[accepted]{5 June 2009}

%%
%% This command processes the author and affiliation and title
%% information and builds the first part of the formatted document.
\maketitle

\section{Introduction}
The goal is to synthesize test input generators that satisfy a given specification. Doing so through overapproximation logic leads to sound but incomplete generators. Such generators could always generate the empty list which would satisfy most conditions one is interested in (e.g.~``sorted'', ``unique'' etc.) but fail to provide a variety of tests that cover the majority of or the entire input space.

Our approach is to use under-approximate logic which ensures completeness but might produce test inputs that do not satisfy the specification. As an initial approach, we use test input filtering, which one still would have to do if using a ``generic'' test generator. Eventually we would like to combine under- and over-approximate logic to produce sound and complete generators that don't need filtering.

\section{Context}

Property Based Testing is useful because it lifts the burden of writing tests by providing a safety specification and a method of generating random inputs. However, non-trivial preconditions can lead to many spurious inputs from generic generators (e.g.~syntactic generators that simply recurse over the constructors but do not consider the semantics), increasing testing time or leading to incomplete/partial coverage. An alternative is hand-writing specialized test input generators which is laborious and potentially faulty. To overcome these limitations, recent work has proposed: fuzzing (and mutation-based) generators, type-based generators, automated Correct-by-Construction generators for limited domains, and over-approximate program synthesis.

Coverage Types\cite{Poirot} are a type based interpretation of the recently developed Incorrectness Logic. These types describe the domain of reachable output values from some valid input value. They can be viewed as the inverse of Refinement Types with their corresponding logical interpretation in Hoare Logic. This kind of under-approximate logic have strong use cases in bug-finding and program analysis.

Program Synthesis is the automation of program creation via some kind of input specification. This specification can be in the form of logic or types as used in deductive synthesis or in the form of input/output examples as used by inductive synthesis.
Gap
However, overapproximation logic tends to perform poorly when addressing the issue of completeness[1]. Overapproximate specifications when used in synthesizing generators allow for generators which only produce trivial values. For example, synthesizing a unique lists generator using overapproximation specification may result in a generator that just produces empty or single element lists which trivially satisfy the specification. Such generators are not useful in PBT.

\section{Innovation}

Inductive synthesis can be seen as a form of under-approximate logic: the input/output examples are only a partial specification. In this lens, similar program synthesis techniques should be applicable when used in a type-based, deductive approach that leverages Coverage Types. A Coverage Type specification would provide a completeness guarantee about the test input generator synthesized, for example that all possible unique lists could be generated.

Our contribution is to implement a fully automated synthesizer that leverages coverage types to construct complete test input generators.

We borrow ideas from inductive synthesis in enumerating small program expressions with inferred coverage types. We can then compose useful fragments using non-deterministic choice to build a more complete generator. In doing so, we hope to prioritize synthesized generators which require less filtering than the generic generator. This offers the opportunity to combine both over and under approximate logic in synthesizing the optimal test generator.
Implementation
Input space: The user will give us an under-approximate(Coverage) type signature for the test generator they want synthesized. They can also provide a list of component functions with under-approximate specifications alongside those already provided by Poirot.

Output space: A functional, non-deterministic program in an ocaml-like language with control flow(if/match) and recursion. Is complete with respect to the user-provided under-approximate specification.
Algorithm:
In a bottom-up fashion, enumerate program blocks of up to depth n.
Use Poirot to infer the coverage types of these program blocks.
Attempt to join program blocks together using non-deterministic choice/control flow to increase their coverage.
If the output space is sufficiently covered by one of these new programs(subtyping check between the user specification and the program succeeds) then finish with the program in hand, else we increase the depth by one and restart.
\section{Evaluation}
We will be using the initial benchmark suite of Poirot's Section 6.3 with the goal of evaluating against that paper's usage of an over-approximate synthesizer(Cobalt) followed by type-checking via Poirot. The success criteria is synthesizing hopefully all of the 6 programs from their corresponding under-approximate specifications and having them be complete with respect to said specification on the first attempt.

There is a secondary goal of producing “better” test generators. Heuristics can be applied along two axis here:
Efficient generators: We should prefer generators that are more likely to produce values within the given coverage specification. This leads to less values being filtered out and thus more valid values for testing. This can be done either by sampling or by leveraging overapproximation type inference to set an upper bound on the program.
Simpler generators: As with most synthesis tasks, we generally lean towards synthesizing smaller programs when equivalent. Additionally in the context of test generators, programs with fewer choice operators might be preferred especially if they can be replaced with more expected, deterministic condition expressions.
\section{Timeline}
First check-in: have a straight-forward and inefficient solution that just tries all possible combinations of program blocks to find a union that covers the whole input space.
Final deliverable: Improve the performance and quality of the synthesizer and produce “better” test input generators.

\begin{acks}
    Christopher Nolan for making Inception. Also Patrick totally stole this project from Ashish's research trash bin.
\end{acks}

%%
%% The next two lines define the bibliography style to be used, and
%% the bibliography file.
\bibliographystyle{ACM-Reference-Format}
\bibliography{cobb}


\end{document}
\endinput
