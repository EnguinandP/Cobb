%%
%% This is file `sample-manuscript.tex',
%% generated with the docstrip utility.
%%
%% The original source files were:
%%
%% samples.dtx  (with options: `manuscript')
%%
%% IMPORTANT NOTICE:
%%
%% For the copyright see the source file.
%%
%% Any modified versions of this file must be renamed
%% with new filenames distinct from sample-manuscript.tex.
%%
%% For distribution of the original source see the terms
%% for copying and modification in the file samples.dtx.
%%
%% This generated file may be distributed as long as the
%% original source files, as listed above, are part of the
%% same distribution. (The sources need not necessarily be
%% in the same archive or directory.)
%%
%% Commands for TeXCount
%TC:macro \cite [option:text,text]
%TC:macro \citep [option:text,text]
%TC:macro \citet [option:text,text]
%TC:envir table 0 1
%TC:envir table* 0 1
%TC:envir tabular [ignore] word
%TC:envir displaymath 0 word
%TC:envir math 0 word
%TC:envir comment 0 0
%%
%%
%% The first command in your LaTeX source must be the \documentclass command.
%%%% Small single column format, used for CIE, CSUR, DTRAP, JACM, JDIQ, JEA, JERIC, JETC, PACMCGIT, TAAS, TACCESS, TACO, TALG, TALLIP (formerly TALIP), TCPS, TDSCI, TEAC, TECS, TELO, THRI, TIIS, TIOT, TISSEC, TIST, TKDD, TMIS, TOCE, TOCHI, TOCL, TOCS, TOCT, TODAES, TODS, TOIS, TOIT, TOMACS, TOMM (formerly TOMCCAP), TOMPECS, TOMS, TOPC, TOPLAS, TOPS, TOS, TOSEM, TOSN, TQC, TRETS, TSAS, TSC, TSLP, TWEB.
% \documentclass[acmsmall]{acmart}

%%%% Large single column format, used for IMWUT, JOCCH, PACMPL, POMACS, TAP, PACMHCI
% \documentclass[acmlarge,screen]{acmart}

%%%% Large double column format, used for TOG
% \documentclass[acmtog, authorversion]{acmart}

%%%% Generic manuscript mode, required for submission
%%%% and peer review
\documentclass[review, sigplan]{acmart}
%% Fonts used in the template cannot be substituted; margin
%% adjustments are not allowed.
%%
%% \BibTeX command to typeset BibTeX logo in the docs
%\AtBeginDocument{%
%  \providecommand\BibTeX{{%
%    \normalfont B\kern-0.5em{\scshape i\kern-0.25em b}\kern-0.8em\TeX}}}

%% Rights management information.  This information is sent to you
%% when you complete the rights form.  These commands have SAMPLE
%% values in them; it is your responsibility as an author to replace
%% the commands and values with those provided to you when you
%% complete the rights form.
%\setcopyright{acmcopyright}
%\copyrightyear{2018}
%\acmYear{2018}
%\acmDOI{XXXXXXX.XXXXXXX}

%% These commands are for a PROCEEDINGS abstract or paper.
%\acmConference[Conference acronym 'XX]{Make sure to enter the correct
%  conference title from your rights confirmation emai}{June 03--05,
%  2018}{Woodstock, NY}
%
%  Uncomment \acmBooktitle if th title of the proceedings is different
%  from ``Proceedings of ...''!
%
%\acmBooktitle{Woodstock '18: ACM Symposium on Neural Gaze Detection,
% June 03--05, 2018, Woodstock, NY}
%\acmPrice{15.00}
%\acmISBN{978-1-4503-XXXX-X/18/06}


%%
%% Submission ID.
%% Use this when submitting an article to a sponsored event. You'll
%% receive a unique submission ID from the organizers
%% of the event, and this ID should be used as the parameter to this command.
%%\acmSubmissionID{123-A56-BU3}

%%
%% For managing citations, it is recommended to use bibliography
%% files in BibTeX format.
%%
%% You can then either use BibTeX with the ACM-Reference-Format style,
%% or BibLaTeX with the acmnumeric or acmauthoryear sytles, that include
%% support for advanced citation of software artefact from the
%% biblatex-software package, also separately available on CTAN.
%%
%% Look at the sample-*-biblatex.tex files for templates showcasing
%% the biblatex styles.
%%

%%
%% The majority of ACM publications use numbered citations and
%% references.  The command \citestyle{authoryear} switches to the
%% "author year" style.
%%
%% If you are preparing content for an event
%% sponsored by ACM SIGGRAPH, you must use the "author year" style of
%% citations and references.
%% Uncommenting
%% the next command will enable that style.
%%\citestyle{acmauthoryear}

%%
%% end of the preamble, start of the body of the document source.
\begin{document}

%%
%% The "title" command has an optional parameter,
%% allowing the author to define a "short title" to be used in page headers.
\title{Cobb: Synthesis of Test Input Generators with Coverage Types}

%%
%% The "author" command and its associated commands are used to define
%% the authors and their affiliations.
%% Of note is the shared affiliation of the first two authors, and the
%% "authornote" and "authornotemark" commands
%% used to denote shared contribution to the research.
\author{Patrick LaFontaine}
\author{Anxhelo Xhebraj}
\author{David Deng}
%\email{}
%\affiliation{%
%    \institution{Purdue University}
%    \streetaddress{}
%    \city{}
%    \state{}
%    \country{USA}
%    \postcode{}
%}


%%
%% By default, the full list of authors will be used in the page
%% headers. Often, this list is too long, and will overlap
%% other information printed in the page headers. This command allows
%% the author to define a more concise list
%% of authors' names for this purpose.
\renewcommand{\shortauthors}{LaFontaine et al.}


%\begin{abstract}
%    This thing is cool
%\end{abstract}

%%
%% The code below is generated by the tool at http://dl.acm.org/ccs.cfm.
%% Please copy and paste the code instead of the example below.
%%
%\begin{CCSXML}

%\end{CCSXML}



%%
%% Keywords. The author(s) should pick words that accurately describe
%% the work being presented. Separate the keywords with commas.
%\keywords{Do, Not, Us, This, Code, Put, the, Correct, Terms, for, Your, Paper}


%\received{20 February 2007}
%\received[revised]{12 March 2009}
%\received[accepted]{5 June 2009}

%%
%% This command processes the author and affiliation and title
%% information and builds the first part of the formatted document.
\maketitle

\section{Introduction}
% Describe the high-level problem you were tackling

Synthesizing test input generators that satisfy a given coverage specification.

Addressing a gap left between type-theoretic but over-approximate synthesis
techniques and

\section{Overview}
% Recap any relevant existing work on this problem, give a high-level
% description of your solution, and summarize the relationship of how your
% solution relates to previous work. Include concrete examples(if applicable),
% that illustrate your solution

\subsection{Property-Based Testing}

\subsection{Bottom-Up Synthesis}

\subsection{Coverage Types}

Coverage Types\cite{Poirot}

\subsection{Our Solution}

Coverage + bottom up + test input generators

Deductive synthesis with recursive specifications + bottom up

The intuition behind blocks, covering subsections of the program space. Small
depth programs as contrasted by long traces needed in inductive synthesis.

\section{Implementation}
% Provide key technical details about your solution, e.g. your toplevel
% algorithm, aprecise encoding of your problem in logic, etc.
We have implemented our solution as a bottom-up deductive synthesizer called
Cobb~(https://github.com/Pat-Lafon/Cobb)

Load in various files

Prepare recursive call, structurally decreasing argument with context

Set typechecker global state, initialize starting seeds: constants, variables,
0-arity functions and constructors, and unit generators. Setting up component
abstractions.

Converting seeds to blocks, anormal form and the use of typing contexts to carry
incremental typing state.

Initializing a collection of blocks based on typing. Old and new maps

Incrementing/joining blocks together. Joining contexts/renaming, false coverage
types. Coverage equivalence as a parallel to observational equivalence

Iterating to a fixed depth

Splitting blocks into subtypes versus supertypes, computing joins over the
subtypes.

Choosing the tightest solution

extracting out the final program

\section{Summary of Results}
% Describe how you validated your approach: if you ran experiments to evaluate
% your approach. provide any data or results from those.

\section{Reflection}
% Describe any key design and implementation challenges; how you addressed them
% (what worked, what didn't, and why);and how this work could lead to a real
% tool or a full-length conference paper. What did you learn from doing this project?

Challenges with building on top of the hodge-podge of Poirot's code base.

Finding the right interactions between over and under approximate reasoning. Not
needing overapproximate subtyping to set an upper bounds. Needing blocks with
both over and under approximate types to account for the hybrid typing of arrow
types.

Unlike traditional bottom-up enumeration, you can't look for a solution at every
iteration. The simplest solution is probably the smallest one though that one will
be too general. For example, if you have the generic generator for the target
type, you have a solution to synthesis before you even start. We also need to
limit the number to SMT queries made as they begin to pile up and are expensive.
It seems like this should be easily parallelizable, especially with the release
of Ocaml 5.0 if the type checker has limited mutable state.

The original need for simpler generators arose out of not knowing enough about
the problem space. What actually needed addressing is when to partition the
input space.

Anything else?

\section{Teamwork}
% A one-paragraph description of the individual team member's contributions

\subsection{Patrick}

\subsection{Anxhelo}

\subsection{David}

\section{Course Topics}
% A one-paragraph description of the course topics applied in the project
This project, through Poirot, makes heavy use of the SMT solver Z3. The
predicates of our type system are embedded in EUFA which is a decidable fragment
of first order logic. The principles of safety specifications and coverage
exploration share parallels with some topics covered like transition system
safety properties and exhaustive state space exploration.

% and a one-paragraph description (if applicable) of any topics that would have
% been useful but weren't covered in the course.
Program synthesis was a planned topic for this course but unfortunately not
covered due to time constraints. In particular, the subtopics of Component-Based
Synthesis and Deductive Synthesis would have been particularly relevant. For
example, covering Synquid or related Haskell synthesis projects. This course
could have gone further in discussing refinement types as a type-theoretic
approach to a decidable fragment of first order logic.

\begin{acks}
    Christopher Nolan for making Inception for whom's main character this tool is
    named after.
    Ashish Mishra who originally purposed the idea of a bottom-up synthesizer
    for coverage types and for allowing Patrick to sit in on his
    under-approximate synthesis meetings and under-approximate reading group.
    Zhe Zhou for the Poirot type checker and showing much patience in answering
    our questions about Poirot's implementation and Coverage Types in general.
\end{acks}

%%
%% The next two lines define the bibliography style to be used, and
%% the bibliography file.
\bibliographystyle{ACM-Reference-Format}
\bibliography{cobb}

\end{document}
\endinput
