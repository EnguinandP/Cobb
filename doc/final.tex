%%
%% This is file `sample-manuscript.tex',
%% generated with the docstrip utility.
%%
%% The original source files were:
%%
%% samples.dtx  (with options: `manuscript')
%%
%% IMPORTANT NOTICE:
%%
%% For the copyright see the source file.
%%
%% Any modified versions of this file must be renamed
%% with new filenames distinct from sample-manuscript.tex.
%%
%% For distribution of the original source see the terms
%% for copying and modification in the file samples.dtx.
%%
%% This generated file may be distributed as long as the
%% original source files, as listed above, are part of the
%% same distribution. (The sources need not necessarily be
%% in the same archive or directory.)
%%
%% Commands for TeXCount
%TC:macro \cite [option:text,text]
%TC:macro \citep [option:text,text]
%TC:macro \citet [option:text,text]
%TC:envir table 0 1
%TC:envir table* 0 1
%TC:envir tabular [ignore] word
%TC:envir displaymath 0 word
%TC:envir math 0 word
%TC:envir comment 0 0
%%
%%
%% The first command in your LaTeX source must be the \documentclass command.
%%%% Small single column format, used for CIE, CSUR, DTRAP, JACM, JDIQ, JEA, JERIC, JETC, PACMCGIT, TAAS, TACCESS, TACO, TALG, TALLIP (formerly TALIP), TCPS, TDSCI, TEAC, TECS, TELO, THRI, TIIS, TIOT, TISSEC, TIST, TKDD, TMIS, TOCE, TOCHI, TOCL, TOCS, TOCT, TODAES, TODS, TOIS, TOIT, TOMACS, TOMM (formerly TOMCCAP), TOMPECS, TOMS, TOPC, TOPLAS, TOPS, TOS, TOSEM, TOSN, TQC, TRETS, TSAS, TSC, TSLP, TWEB.
% \documentclass[acmsmall]{acmart}

%%%% Large single column format, used for IMWUT, JOCCH, PACMPL, POMACS, TAP, PACMHCI
% \documentclass[acmlarge,screen]{acmart}

%%%% Large double column format, used for TOG
% \documentclass[acmtog, authorversion]{acmart}

%%%% Generic manuscript mode, required for submission
%%%% and peer review
\documentclass[review, sigplan]{acmart}
\usepackage{listings}
\usepackage{cleveref}
%% Fonts used in the template cannot be substituted; margin
%% adjustments are not allowed.
%%
%% \BibTeX command to typeset BibTeX logo in the docs
%\AtBeginDocument{%
%  \providecommand\BibTeX{{%
%    \normalfont B\kern-0.5em{\scshape i\kern-0.25em b}\kern-0.8em\TeX}}}

%% Rights management information.  This information is sent to you
%% when you complete the rights form.  These commands have SAMPLE
%% values in them; it is your responsibility as an author to replace
%% the commands and values with those provided to you when you
%% complete the rights form.
%\setcopyright{acmcopyright}
%\copyrightyear{2018}
%\acmYear{2018}
%\acmDOI{XXXXXXX.XXXXXXX}

%% These commands are for a PROCEEDINGS abstract or paper.
%\acmConference[Conference acronym 'XX]{Make sure to enter the correct
%  conference title from your rights confirmation emai}{June 03--05,
%  2018}{Woodstock, NY}
%
%  Uncomment \acmBooktitle if th title of the proceedings is different
%  from ``Proceedings of ...''!
%
%\acmBooktitle{Woodstock '18: ACM Symposium on Neural Gaze Detection,
% June 03--05, 2018, Woodstock, NY}
%\acmPrice{15.00}
%\acmISBN{978-1-4503-XXXX-X/18/06}


%%
%% Submission ID.
%% Use this when submitting an article to a sponsored event. You'll
%% receive a unique submission ID from the organizers
%% of the event, and this ID should be used as the parameter to this command.
%%\acmSubmissionID{123-A56-BU3}

%%
%% For managing citations, it is recommended to use bibliography
%% files in BibTeX format.
%%
%% You can then either use BibTeX with the ACM-Reference-Format style,
%% or BibLaTeX with the acmnumeric or acmauthoryear sytles, that include
%% support for advanced citation of software artefact from the
%% biblatex-software package, also separately available on CTAN.
%%
%% Look at the sample-*-biblatex.tex files for templates showcasing
%% the biblatex styles.
%%

%%
%% The majority of ACM publications use numbered citations and
%% references.  The command \citestyle{authoryear} switches to the
%% "author year" style.
%%
%% If you are preparing content for an event
%% sponsored by ACM SIGGRAPH, you must use the "author year" style of
%% citations and references.
%% Uncommenting
%% the next command will enable that style.
%%\citestyle{acmauthoryear}

%%
%% end of the preamble, start of the body of the document source.
\begin{document}

%%
%% The "title" command has an optional parameter,
%% allowing the author to define a "short title" to be used in page headers.
\title{Cobb: Synthesis of Test Input Generators with Coverage Types}

%%
%% The "author" command and its associated commands are used to define
%% the authors and their affiliations.
%% Of note is the shared affiliation of the first two authors, and the
%% "authornote" and "authornotemark" commands
%% used to denote shared contribution to the research.
\author{Patrick LaFontaine}
\author{Anxhelo Xhebraj}
\author{David Deng}
%\email{}
%\affiliation{%
%    \institution{Purdue University}
%    \streetaddress{}
%    \city{}
%    \state{}
%    \country{USA}
%    \postcode{}
%}


%%
%% By default, the full list of authors will be used in the page
%% headers. Often, this list is too long, and will overlap
%% other information printed in the page headers. This command allows
%% the author to define a more concise list
%% of authors' names for this purpose.
\renewcommand{\shortauthors}{LaFontaine et al.}

\newcommand\todo[1]{\textcolor{red}{#1}}

%\begin{abstract}
%    This thing is cool
%\end{abstract}

%%
%% The code below is generated by the tool at http://dl.acm.org/ccs.cfm.
%% Please copy and paste the code instead of the example below.
%%
%\begin{CCSXML}

%\end{CCSXML}



%%
%% Keywords. The author(s) should pick words that accurately describe
%% the work being presented. Separate the keywords with commas.
%\keywords{Do, Not, Us, This, Code, Put, the, Correct, Terms, for, Your, Paper}


%\received{20 February 2007}
%\received[revised]{12 March 2009}
%\received[accepted]{5 June 2009}

%%
%% This command processes the author and affiliation and title
%% information and builds the first part of the formatted document.
\maketitle

\section{Introduction}
% Describe the high-level problem you were tackling

The goal of this project is to utilize coverage types to synthesize test input generators.
Common approaches use an overapproxime synthesis which leads to sound but incomplete generators.
With list generators, for example, generators produced using overapproximate synthesis could always generate the empty list which would satisfy most conditions one is interested in (e.g.~``sorted'', ``unique'' etc.) but fail to provide a variety of tests that cover the majority of or the entire input space.

Instead of overapproximate synthesis that uses refinement types as a basis for test input specification, our approach uses coverage types\citep{Poirot}, which is based on a recently proposed underapproximate, incorrectness logic\citep{IL}, to ensure the completeness of the synthesized test input generators.
% On the flip side, this approach but might produce extraneous test inputs that fall outside of the given specification.

\section{Overview}
% Recap any relevant existing work on this problem, give a high-level
% description of your solution, and summarize the relationship of how your
% solution relates to previous work. Include concrete examples(if applicable),
% that illustrate your solution

\subsection{Property-Based Testing}

Property Based Testing (PBT) stands in the middle ground between random testing and formal verification, in that the input generation is guided by the interested property, specified by programmers as logical predicates. This allows the programmers to guide the generated input using the testing predicate, allowing it to discover more intricate bugs, at the same time leveraging automation using SMT solvers to alleviate manual effort.

Property Based Testing  is useful because it lifts the burden of writing tests by providing a safety specification and a method of generating random inputs.
However, non-trivial preconditions can lead to many spurious inputs from generic generators (e.g.~syntactic generators that simply recurse over the constructors but do not consider the semantics), increasing testing time or leading to incomplete/partial coverage.
An alternative is hand-writing specialized test input generators which is laborious and potentially faulty.
To overcome these limitations, recent work has proposed: fuzzing (and mutation-based) generators, type-based generators, automated Correct-by-Construction generators for limited domains, and over-approximate program synthesis.

\subsection{Coverage Types}

Coverage Types~\cite{Poirot} are a type based interpretation of the recently developed Incorrectness Logic\citep{IL}.
These types describe the domain of reachable output values from some valid input value.
They can be viewed as the inverse of Refinement Types with their corresponding logical interpretation in Hoare Logic.
This kind of under-approximate logic have strong use cases in bug-finding and program analysis.

\subsection{Bottom-Up Synthesis}
We are taking ideas from bottom-up synthesis by starting from the leaf node.
Although bottom-up synthesis (citation?) are often inductive and example-based, we use a deductive approach that is based on underapproximate type-checking to ensure correctness of the synthesized program.

\subsection{Deductive synthesis}

Program Synthesis makes use of input specification in the form of logic or types as used in deductive synthesis or in the form of input/output examples as used by inductive synthesis.
Gap
However, overapproximation logic tends to perform poorly when addressing the issue of completeness. Overapproximate specifications when used in synthesizing generators allow for generators which only produce trivial values. For example, synthesizing a unique lists generator using overapproximation specification may result in a generator that just produces empty or single element lists which trivially satisfy the specification. Such generators are not useful in PBT.

\subsection{Our Solution}

Coverage + bottom up + test input generators

Deductive synthesis with recursive specifications + bottom up

The intuition behind blocks, covering subsections of the program space. Small
depth programs as contrasted by long traces needed in inductive synthesis.

\section{Implementation}
% Provide key technical details about your solution, e.g. your toplevel
% algorithm, aprecise encoding of your problem in logic, etc.
We have implemented our solution as a bottom-up deductive synthesizer called
Cobb~(\url{https://github.com/Pat-Lafon/Cobb}).
Internally the algorithm uses Poirot for guiding the synthesis and pruning
the search space.

The program requires the user to specify the refined signature of the function
they would like to synthesize and a set of ``components'' and types
that should be used for enumerating terms.
Note that the components should have their refined types specified as
well.

An example of the initial information that the user provides
is shown below.

\begin{align*}
   & T := \textsf{bool} \mid \textsf{int} \mid \textsf{int list}                                                            \\
   & \text{Seed} := \textsf{bool\_gen}() \mid 0 \mid 1 \mid \textsf{int\_gen()} \mid \textsf{nil} \mid \textsf{list\_gen()} \\
   & \text{Component} :=                                                                                                    \\
   & \  \mid +: \textsf{int} \rightarrow \textsf{int} \rightarrow \textsf{int}                                              \\
   & \  \mid -: \textsf{int} \rightarrow \textsf{int} \rightarrow \textsf{int}                                              \\
   & \  \mid \leq: \textsf{int} \rightarrow \textsf{int} \rightarrow \textsf{bool}                                          \\
   & \  \mid \textsf{cons}: \textsf{int} \rightarrow \textsf{int list}                                                      \\
   & \dots
\end{align*}
The types that should be used in the synthesis are
booleans, integers and lists of integers.
Seeds are elementary constructors such as 0, 1, and \textsf{nil}
but also non-deterministic applications of generators of values of
base types.

Components are functions or constructors.
All elements needed for the synthesis must have all the signatures
that are needed for Poirot to be able to check them.
Fortunately, by leveraging Poirot we reduce the overhead needed for setting
up the synthesis by reusing key lemmas and properties of common
data-types provided by its codebase already.

\todo{Load in various files}

Cobb additionally supports the generation of recursive functions.
We achieve this by adding the signature of the function being
generated in the pool of components available for use,
no different from standard typing of recursive bindings in non-refined
type-systems.
However, an additional constraint is added to the return type
of the function's specification to ensure that only applications
where at least one argument is structurally decreasing are formed.

\todo{Set typechecker global state, initialize starting seeds: constants, variables,
  0-arity functions and constructors, and unit generators. Setting up component
  abstractions.
}

Once seeds, components, and self-application with structurally decreasing
arguments is setup, the synthesis algorithm begins by enumerating
all terms up to a target level\footnote{The level of a node is what would
  be commonly referred to as height in the tree. But since
  we are constructing trees bottom-up we prefer the term level} specified by the user.
\begin{lstlisting}[language=Python, basicstyle=\small\ttfamily, mathescape, numbers=left, numbersep=3pt]
def synth(seeds, components, tgt_sign, tgt_l):
 blocks = seeds
 for l in range(tgt_l):
   current_level = apply(components, blocks)
   # Prune redundant programs
   for p1, p2 in product(current_level):
     if coverage(p1) == coverage(p2): $\label{line:cvgcheck}$
       current_level.remove(p1)
   blocks = current_level + blocks

 return join(blocks, tgt_sign)
\end{lstlisting}
At level $\ell + 1$, we enumerate all blocks of height $\ell + 1$ by applying
the components to the blocks of level $i \leq \ell$.
Accordingly, at least one of the arguments must be of height $\ell$ to produce
a tree of height $\ell + 1$.
When building new terms in \lstinline[basicstyle=\small\ttfamily]|apply|
we ensure that components are applied only to \emph{simply well-typed} terms
to avoid expensive SMT calls for trivially simply ill-typed terms.

\begin{lstlisting}[language=Python, basicstyle=\small\ttfamily, mathescape, numbers=left, numbersep=3pt]
def apply(components, blocks):
 for c in components:
   argss = product_filter(blocks, c.arg_types)
   for args in argss:
      block = $'${ c(*args) }
      # Skip block if covered by argument
      if any(coverage(block) $\leq$ coverage(arg) $\label{line:cvginc}$
            for arg in args):
        continue
      yield block
\end{lstlisting}

Therefore, \lstinline[basicstyle=\small\ttfamily]|product_filter|
generates all potential arguments sequences \lstinline[basicstyle=\small\ttfamily]|argss|
that a component can be applied to, to produce a new simply well-typed
term of height $\ell + 1$.

\subsection{Pruning the search space}
There are two key optimizations we perform:
(1) when building a new block in \lstinline[basicstyle=\small\ttfamily]|apply|,
we make sure that it has an increased coverage with respect to its arguments (\Cref{line:cvgcheck}).
This is needed to avoid spurious terms such as
\lstinline[basicstyle=\small\ttfamily]|int_gen() + int_gen()|.
(2) once the blocks at level $\ell + 1$ are enumerated,
we prune blocks that have equivalent coverage to other blocks at the same
level.
This check approximates observational equivalence and deduplicates
blocks of the form \lstinline[basicstyle=\small\ttfamily]|a + 1|
and \lstinline[basicstyle=\small\ttfamily]|1 + a| which are
syntactically different but observationally equivalent.

\subsection{Adding control-flow}

The algorithm we described so far builds only blocks without
control-flow.
However control-flow is necessary to build any non-trivial term and
especially so in the presence of over-approximate specifications
in the parameters of the functions.

Consider for exmaple the component
\lstinline[basicstyle=\small\ttfamily, mathescape]|f: ${\color{red}\{} s: \mathsf{int} \mid s \geq 0 {\color{red}\}} \rightarrow \dots $|
which requires for the first argument to be a natural and suppose
we would like to apply that component to \lstinline[basicstyle=\small\ttfamily, mathescape]|x : $\mathsf{int}$|.
Simply creating the term \lstinline[language=caml, basicstyle=\small\ttfamily, mathescape]|f x|
would lead to an unreachable execution where the return type is $\bot$.
Whenever such cases occur, we try adding control-flow that can potentially
satisfy the unmet precondition.
We create the term \lstinline[language=caml, basicstyle=\small\ttfamily, mathescape]|if $(b: \textsf{bool})$ then f x else Exn|
where $b$ can be \emph{any} block where the base type is \textsf{bool}
and check if that leads to well-typed term\footnote{
As a potential optimization we could try only boolean blocks where
variables appearing in the over-approximate signature appear however
that complicates implementation and could reduce completeness of the
enumeration.}.
We considered the alternative of treating conditionals as components
$\textsf{if}_\textsf{T} \rightarrow \textsf{bool} \rightarrow \textsf{T} \rightarrow \textsf{T}$
however we've found that the increase in search space size
bottlenecks exploration.
Our strategy allows to build terms such as
\lstinline[language=caml, basicstyle=\small\ttfamily, mathescape]|if x $\geq$ 0 then f x else Exn|
increasing the search space \emph{on-demand} only when required
by the function signatures.

\subsection{Finding the Best Program That Meets the Specification}
Once we have enumerated all ASTs up to a given target level the
final missing piece is which of the synthesized program
meets the target specification the best.
First, we can note that if a ``generic'' seed that generates
all the space of values inhabited by the target specification is provided
(e.g. \lstinline[language=caml, basicstyle=\small\ttfamily, mathescape]|list_gen()|)
then the synthesis algorithm always succeeds in finding a program satisfying the
target under-approximation specification.
However such program is not a satisfying synthesis in that it doesn't satisfy
the over-approximation specification as well.

To generate a target program that satisfies the over-approximation
``as close as possible'' we use the following heuristic.
We partition the blocks into four sets:
\lstinline[language=caml, basicstyle=\small\ttfamily, mathescape]|none|,
the programs that can be discarded since they do not satisfy the target
specification;
\lstinline[language=caml, basicstyle=\small\ttfamily, mathescape]|eq|,
the programs that are exactly equivalent to the specification which
means that they also satisfy the over-approximate specification and
would be the ideal candidate;
\lstinline[language=caml, basicstyle=\small\ttfamily, mathescape]|sub|,
programs that have smaller coverage than the target specification;
\lstinline[language=caml, basicstyle=\small\ttfamily, mathescape]|sup|,
programs that have bigger coverage than the target specification
(i.e. contains \lstinline[language=caml, basicstyle=\small\ttfamily, mathescape]|list_gen()|).

\begin{lstlisting}[language=Python, basicstyle=\small\ttfamily, mathescape, numbers=left, numbersep=3pt]
def join(blocks, target):
 none, sub, eq, sup = (
    partition(blocks, target)
 )
 if len(eq) > 0:
   yield eq

 for b1, b2 in product(sub):
   candidate = $'${if bool_gen() then b1 else b2}
   if target $\leq$ coverage(candidate):
     yield candidate

 if len(sup) > 0:
   yield sup[0]
 yield None
\end{lstlisting}

If \lstinline[language=caml, basicstyle=\small\ttfamily, mathescape]|eq| is non-empty,
then we are done and return the first program in such set. It could be possible
to also select the smallest among such programs but we don't do that as of the moment.
If there is no program coverage-equivalent to the specification, then
we try to non-deterministically join the programs
in \lstinline[language=caml, basicstyle=\small\ttfamily, mathescape]|sub| to meet
the specification.
The key idea here is that there could be programs that don't meet
the specification just because of a single case but their join
could meet it.
Finally, as last resort we return a program that surely over-covers
the target specification.

\subsection{Representing AST Nodes for Typing}
\todo{
  Converting seeds to blocks, anormal form and the use of typing contexts to carry
  incremental typing state.
}

Similarly to other refinement-type checkers, Poirot uses
ANF to simplify checking of dependencies of return types on arguments.
The standard approach of bottom-up synthesis to work on ASTs would
be suboptimal -- inferring the types of new terms would require to
first translate them in ANF and then for the type-checker to traverse
the ANF term to ``unpack'' it into contexts while inferring intermediate
nodes to finally type the full tree.

Our solution instead relies on representing terms by their
typing contexts and a global term table that keeps track of
the term corresponding to each variable.
Consider the following ANF term.
\begin{lstlisting}[language=caml, basicstyle=\small\ttfamily]
let a = int_gen()
 let b = 2
   a + b
\end{lstlisting}
The only information we keep track of is its context

\begin{lstlisting}[language=caml, basicstyle=\small\ttfamily, mathescape]
{
 a: ${\color{blue}[}\nu: \textsf{int} \mid \textit{true}{\color{blue}]}$,
 b: ${\color{blue}[}\nu: \textsf{int} \mid \nu == 2{\color{blue}]}$
 c: ${\color{blue}[}\nu: \textsf{int} \mid \textit{true}{\color{blue}]}$
}
\end{lstlisting}
where \lstinline|c| is the type of the value resulting from the
ANF term.


Initializing a collection of blocks based on typing. Old and new maps

Incrementing/joining blocks together. Joining contexts/renaming, false coverage
types. Coverage equivalence as a parallel to observational equivalence

Iterating to a fixed depth

Splitting blocks into subtypes versus supertypes, computing joins over the
subtypes.

Choosing the tightest solution

extracting out the final program

\section{Summary of Results}
% Describe how you validated your approach: if you ran experiments to evaluate
% your approach. provide any data or results from those.

\section{Reflection}
% Describe any key design and implementation challenges; how you addressed them
% (what worked, what didn't, and why);and how this work could lead to a real
% tool or a full-length conference paper. What did you learn from doing this
% project?

\subsection{Poirot as a black-box}
For this project, we created a fork
\footnote{\url{https://github.com/Pat-Lafon/underapproximation_type}} of Poirot
\footnote{\url{https://github.com/zhezhouzz/underapproximation_type/commit/beb65e3410d07f207f5b75a09a7428c146b4099f}}
with only minor modifications to help print out values and dispatch to the
correct subtyping calls for \lstinline|MMT.t| types. In general, learning how to
navigate this code base took some getting use to for our team. Our synthesizer
largely uses Poirot as a black-box to dispatch subtyping checks. This choice was
largely over our lack of confidence with modifying the typechecking procedure
for fear of introducing bugs. Alternatively, a more tightly coupled integration
between the type checker and synthesizer could have a lot of benefit. Since
there are multiple variations of a term with small changes to the argument,
there are many duplicates or structurally similar subtyping checks that could
be reused.

\subsection{Challenges with using an under-approximate logic for synthesis}
In traditional bottom-up synthesis, you enumerate a set of terms at a given
level and then check for a solution. If one isn't found, you repeat the process
for the next level. Our solution instead only checks for a solution at the very
end after enumerating to a user-specified depth. This is done for two reasons.
For one, the process of checking for a solution is currently expensive as we
perform a subtyping check on each term of the correct base type to partition the
solution space, perform SMT queries to find an optimal set of smaller terms to
join together, and then a final round of subtyping checks to identify a
solution. This procedure scales poorly as the number of terms increases and as
such should only really be done once. It would be interesting to consider a way
to do this incrementally.

Additionally, because the logic is under-approximate, the traditional benefit of
a deductive synthesis search does not apply as there are most likely many
possible solutions. And unlike in traditional inductive synthesis, the smallest
solution is unlikely to be the best one because it is probably too general. For
example, if the generic generator for a given base type is available, then it
will be included as a seed term and will always be a solution at any depth
level. This means that the synthesis procedure can not attempt to terminate
early as a better solution may be available. It is possible that heuristics

\subsection{Leveraging both over- and under-approximate logics}
Finding the right interactions between over and under approximate reasoning. Not
needing over-approximate subtyping to set an upper bounds. Needing blocks with
both over and under approximate types to account for the hybrid typing of arrow
types.

The original need for simpler generators arose out of not knowing enough about
the problem space. What actually needed addressing is when to partition the
input space.

\subsection{CBS with non-determinism}

\subsection{If as a component}

\subsection{Abduction}

\subsection{Parallelism}
It seems like this should be easily parallelizable, especially with the release
of Ocaml 5.0 if the type checker has limited mutable state.

\subsection{Possible Typechecker Bugs?}
In our project, the use of Poirot was largely treated as a black-box


In the
course of using Poirot, we believe we may have run into some issues that
impacted the correctness of our synthesizer.

\subsubsection{Let binding immediate shadowing}

\subsubsection{Measure issues? The singleton list}

\subsubsection{Structural decreasing constraint on the return type instead of the argument}


\section{Teamwork}
% A one-paragraph description of the individual team member's contributions

\subsection{Patrick}

\subsection{Anxhelo}
\begin{itemize}
  \item Make environment automated and pin dependencies to get started quickly
  \item Design managing terms as contexts (jointly with Patrick) and implement
    related utilities
  \item Implement seed and component discovery and abstraction
  \item Abstract Poirot terms in a simpler interface for synthesis
  \item Sketch and debug substitution (jointly with David)
  \item Add support for global term table and final program printing
  \item Refine and fix bugs in subtyping checks
  \item Criticizing Patrick's term exploration decisions
\end{itemize}

\subsection{David}
\begin{itemize}
  \item Understand the synthesis algorithm with zero prior experience in Ocaml
  \item Implement substitution helper functions and refactoring
  \item Understand the challenges we faced with the explanation of Patrick and Anxhelo
  \item Help finding high resolution memes for the presentation
\end{itemize}

\section{Course Topics}
% A one-paragraph description of the course topics applied in the project
This project, through Poirot, makes heavy use of the SMT solver Z3. The
predicates of our type system are embedded in EUFA which is a decidable fragment
of first order logic. The principles of safety specifications and coverage
exploration share parallels with some topics covered like transition system
safety properties and exhaustive state space exploration.

% and a one-paragraph description (if applicable) of any topics that would have
% been useful but weren't covered in the course.
Program synthesis was a planned topic for this course but unfortunately not
covered due to time constraints. In particular, the subtopics of Component-Based
Synthesis and Deductive Synthesis would have been particularly relevant. For
example, covering Synquid or related Haskell synthesis projects. This course
could have gone further in discussing refinement types as a type-theoretic
approach to a decidable fragment of first order logic.

\begin{acks}
  Christopher Nolan for making Inception for whom's main character this tool is
  named after.
  Ashish Mishra who originally purposed the idea of a bottom-up synthesizer
  for coverage types and for allowing Patrick to sit in on his
  under-approximate synthesis meetings and under-approximate reading group.
  Zhe Zhou for the Poirot type checker and showing much patience in answering
  our questions about Poirot's implementation and Coverage Types in general.
\end{acks}

%%
%% The next two lines define the bibliography style to be used, and
%% the bibliography file.
\bibliographystyle{ACM-Reference-Format}
\bibliography{cobb}

\end{document}
\endinput
